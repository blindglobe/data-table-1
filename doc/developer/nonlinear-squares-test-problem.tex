
\documentclass{article}
%%%%%%%%%%%%%%%%%%%%%%%%%%%%%%%%%%%%%%%%%%%%%%%%%%%%%%%%%%%%%%%%%%%%%%%%%%%%%%%%%%%%%%%%%%%%%%%%%%%%%%%%%%%%%%%%%%%%%%%%%%%%%%%%%%%%%%%%%%%%%%%%%%%%%%%%%%%%%%%%%%%%%%%%%%%%%%%%%%%%%%%%%%%%%%%%%%%%%%%%%%%%%%%%%%%%%%%%%%%%%%%%%%%%%%%%%%%%%%%%%%%%%%%%%%%%
\usepackage{amsfonts}
\usepackage{amsmath}

\setcounter{MaxMatrixCols}{10}
%TCIDATA{OutputFilter=LATEX.DLL}
%TCIDATA{Version=5.50.0.2960}
%TCIDATA{<META NAME="SaveForMode" CONTENT="1">}
%TCIDATA{BibliographyScheme=Manual}
%TCIDATA{Created=Sunday, December 09, 2012 07:04:33}
%TCIDATA{LastRevised=Sunday, January 06, 2013 09:04:03}
%TCIDATA{<META NAME="GraphicsSave" CONTENT="32">}
%TCIDATA{<META NAME="DocumentShell" CONTENT="Standard LaTeX\Standard LaTeX Article">}
%TCIDATA{CSTFile=USTDC-BriefReport.cst}

\newtheorem{theorem}{Theorem}
\newtheorem{acknowledgement}[theorem]{Acknowledgement}
\newtheorem{algorithm}[theorem]{Algorithm}
\newtheorem{axiom}[theorem]{Axiom}
\newtheorem{case}[theorem]{Case}
\newtheorem{claim}[theorem]{Claim}
\newtheorem{conclusion}[theorem]{Conclusion}
\newtheorem{condition}[theorem]{Condition}
\newtheorem{conjecture}[theorem]{Conjecture}
\newtheorem{corollary}[theorem]{Corollary}
\newtheorem{criterion}[theorem]{Criterion}
\newtheorem{definition}[theorem]{Definition}
\newtheorem{example}[theorem]{Example}
\newtheorem{exercise}[theorem]{Exercise}
\newtheorem{lemma}[theorem]{Lemma}
\newtheorem{notation}[theorem]{Notation}
\newtheorem{problem}[theorem]{Problem}
\newtheorem{proposition}[theorem]{Proposition}
\newtheorem{remark}[theorem]{Remark}
\newtheorem{solution}[theorem]{Solution}
\newtheorem{summary}[theorem]{Summary}
\newenvironment{proof}[1][Proof]{\noindent\textbf{#1.} }{\ \rule{0.5em}{0.5em}}
\input{tcilatex}
\begin{document}

\title{Standard 
%TCIMACRO{\TeXButton{LaTeX}{\LaTeX{}} }%
%BeginExpansion
\LaTeX{}
%EndExpansion
Article}
\author{A. U. Thor \\
%EndAName
The University of Stewart Island}
\maketitle

\begin{abstract}
We study the effects of warm water on the local penguin population. The
major finding is that it is extremely difficult to induce penguins to drink
warm water. The success factor is approximately $-e^{-i\pi }-1$.
\end{abstract}

\section{Nonlinear least square fitting}

I define a test function and data for which I can obtain the residual, its
square, and its partial derivatives in closed form.

\subsection{Example data and test function}

For the data I consider the exact function 
\begin{equation}
y_{i}=ae^{-bi}  \label{def:y_i}
\end{equation}%
with 
\begin{equation}
i=0,1,\ldots ,9  \label{def:i:0-9}
\end{equation}%
modeled by the two parameter curve%
\begin{equation}
y=Ae^{-Bi}  \label{def:y(i)}
\end{equation}

\subsection{Residual and its derivative}

The residual is%
\begin{eqnarray}
r_{i} &=&y_{i}-y\left( i\right) \\
&=&ae^{-bi}-Ae^{-Bi}  \label{eq:r_i^2}
\end{eqnarray}%
Its residual with respect to $A$ and $B$ is calculated next. For $A$,%
\begin{eqnarray}
\frac{\partial r_{i}}{\partial A} &=&-\frac{\partial y}{\partial A} \\
&=&-e^{-Bi}
\end{eqnarray}%
and for $B$%
\begin{eqnarray}
\frac{\partial r_{i}}{\partial B} &=&-\frac{\partial y}{\partial B} \\
&=&Aie^{-Bi}
\end{eqnarray}

The residual squared is%
\begin{eqnarray}
r_{i}^{2} &=&\left( ae^{-bi}-Ae^{-Bi}\right) ^{2} \\
&=&a^{2}e^{-2bi}-2aAe^{-\left( b+B\right) i}+A^{2}e^{-2Bi}
\end{eqnarray}%
which is written as%
\begin{equation}
r_{i}^{2}=a^{2}\left( e^{-2b}\right) ^{i}-2aA\left( e^{-\left( b+B\right)
}\right) ^{i}+A^{2}\left( e^{-2B}\right) ^{i}
\end{equation}

Introducint%
\begin{eqnarray}
\alpha &=&e^{-2b}  \label{def:alpha} \\
\beta &=&e^{-\left( b+B\right) }  \label{def:beta} \\
\gamma &=&e^{-2B}  \label{def:gamma}
\end{eqnarray}%
the residual squared is%
\begin{equation}
r_{i}^{2}=a^{2}\alpha ^{i}-2aA\beta ^{i}+A^{2}\gamma ^{i}
\label{eq:r_i^2(alpha,beta,gamma)}
\end{equation}%
and their sum is%
\begin{eqnarray}
S &=&\sum r_{i}^{2} \\
&=&a^{2}\sum \alpha ^{i}-2aA\sum \beta ^{i}+A^{2}\sum \gamma ^{i}
\label{eq:S}
\end{eqnarray}

\subsection{Summation function}

I will be using the summation identity%
\begin{eqnarray}
\sum_{i=0}^{n}c^{i} &=&\frac{1-c^{n+1}}{1-c} \\
&=&\sigma _{n}\left( c\right)  \label{def:sigma-fun}
\end{eqnarray}%
and its derivativeThe derivative of $\sigma $ is%
\begin{eqnarray}
\frac{d\sigma }{dc} &=&\frac{d}{dc}\left( \frac{1-c^{n+1}}{1-c}\right) \\
&=&\frac{\left( 1-c\right) \left( -\left( n+1\right) c^{n}\right) -\left(
1-c^{n+1}\right) \left( -1\right) }{1-c^{2}} \\
&=&\frac{cc^{n}n-c^{n}-c^{n}n+1}{1-c^{2}} \\
&=&\frac{nc^{n+1}-\left( n+1\right) c^{n}+1}{1-c^{2}} \\
&=&\frac{c^{n}\left( n\left( c-1\right) -1\right) +1}{1-c^{2}}
\label{eq:sigma'}
\end{eqnarray}

Using $\sigma $, the sum of residual squared is%
\begin{equation}
S=a^{2}\sigma _{n}\left( \alpha \right) -2aA\sigma \left( \beta \right)
+A^{2}\sigma \left( \gamma \right)  \label{eq:S(sigma,alpha,beta,gamma)}
\end{equation}

\subsection{Partial derivatives of $S$}

I\ now derive the partial derivatives of $S$. I make sure that at match%
\begin{eqnarray}
A &=&a  \label{def:match} \\
B &=&b
\end{eqnarray}%
the partial derivative is zero.

The derivative of $S$ with respect to $A$ is%
\begin{equation}
\frac{\partial S}{\partial A}=-2a\sigma \left( \beta \right) +2A\sigma
\left( \gamma \right)
\end{equation}%
At match, this becomes%
\begin{eqnarray}
\frac{\partial S}{\partial A} &=&a\sigma \left( \beta \right) +2a\sigma
\left( \beta \right) \\
&=&0
\end{eqnarray}

The derivative of $S$ with respect to $B$ is%
\begin{equation}
\frac{\partial S}{\partial B}=-2aA\frac{d\sigma \left( \beta \right) }{%
d\beta }\frac{\partial \beta }{\partial B}+A^{2}\frac{d\sigma \left( \gamma
\right) }{d\gamma }\frac{\partial \gamma }{\partial B}
\end{equation}

The derivatives of $\beta $ and $\gamma $ are%
\begin{eqnarray}
\frac{\partial \beta }{\partial B} &=&-\beta \\
&=&-\gamma \\
\frac{\partial \gamma }{\partial B} &=&-2\gamma
\end{eqnarray}%
The partial derivative is%
\begin{equation}
\frac{\partial S}{\partial B}=2aA\gamma \frac{d\sigma \left( \beta \right) }{%
d\beta }-2A^{2}\gamma \frac{d\sigma \left( \gamma \right) }{d\gamma }
\end{equation}

At match this becomes%
\begin{eqnarray}
\frac{\partial S}{\partial A} &=&2a^{2}\gamma \frac{d\sigma \left( \beta
\right) }{d\beta }-2a^{2}\gamma \frac{d\sigma \left( \beta \right) }{d\beta }
\\
&=&0
\end{eqnarray}

\section{Another function}

\begin{equation}
y=A10^{Bx}
\end{equation}%
the Jacobian is%
\begin{eqnarray}
\frac{d}{dA}\left( A10^{Bx}\right)  &=&10^{Bx} \\
\frac{d}{dB}\left( A10^{Bx}\right)  &=&\ln 10Ax10^{Bx}
\end{eqnarray}

\end{document}
